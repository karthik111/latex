% Richard Klein 2017

% Use texlive-full
% sudo apt install python-pygments

% Include Packages
%\usepackage[a4paper,inner=3.5cm,outer=2.5cm,top=2.5cm,bottom=2.5cm]{geometry}  % Set page margins
\usepackage{fullpage}

%\usepackage[caption=false,font=footnotesize,labelfont=sf,textfont=sf]{subfig}
\usepackage[caption=false,font=footnotesize]{subfig}
\usepackage{float}                  % Allows 'Here and Only Here' [H] for Floats
\usepackage{url}                    % \url{} command
\usepackage{times}                  % Set font to Times
\usepackage{graphicx}               % \includegraphics
%\usepackage{subfigure}             % Allow subfigures
\usepackage{amsmath}				% Useful symbols etc
\usepackage{amssymb}
\usepackage{amsthm}
\usepackage{multicol}				% Table Stuff
\usepackage{multirow}
\usepackage{hhline}
\usepackage{caption}				% Figure/Table Caption stuff
\usepackage{pdfpages}				% Lets you include pdf documents

% Referencing
\usepackage{varioref}               % Provides \Vref and \vref to indicate where a reference is.
\usepackage[bookmarks=true,bookmarksopen=true]{hyperref}               % Hyperlinks references
\usepackage{cleveref}               % Provides \Cref, \cref, \Vref, \vref to include the type of reference: fig/eqn/tbl

\hypersetup{
  colorlinks   = true,              %Colours links instead of ugly boxes
  urlcolor     = blue,              %Colour for external hyperlinks
  linkcolor    = blue,              %Colour of internal links
  citecolor    = blue,                %Colour of citations
  pdftitle={PDF TITLE},%<!CHANGE!
  pdfsubject={PDF SUBJECT},%<!CHANGE!
  pdfauthor={AUTHORE NAME},%<!CHANGE!
}

%%% CleverRef Setups %%%
\crefname{table}{table}{tables}	
\Crefname{table}{Table}{Tables}

\crefname{figure}{figure}{figures}
\Crefname{figure}{Figure}{Figures}

\crefname{equation}{equation}{equations}
\Crefname{equation}{Equation}{Equations}

% Theorems
\newtheorem{theorem}{Theorem}[section]
\newtheorem{lemma}[theorem]{Lemma}
\newtheorem{proposition}[theorem]{Proposition}
\newtheorem{corollary}[theorem]{Corollary}

% Line Spacing
%\usepackage{setspace}
%\linespread{2}

% Spaces beteen paragraphs
%\newcommand{\setparskip}{\setlength{\parskip}{10pt plus10pt minus4pt}}
%\newcommand{\unsetparskip}{\setlength{\parskip}{0pt plus0pt minus0pt}}

%\newenvironment{proof}[1][Proof]{\begin{trivlist}
%\item[\hskip \labelsep {\bfseries #1}]}{\end{trivlist}}
%\newenvironment{definition}[1][Definition]{\begin{trivlist}
%\item[\hskip \labelsep {\bfseries #1}]}{\end{trivlist}}
%\newenvironment{example}[1][Example]{\begin{trivlist}
%\item[\hskip \labelsep {\bfseries #1}]}{\end{trivlist}}
%\newenvironment{remark}[1][Remark]{\begin{trivlist}
%\item[\hskip \labelsep {\bfseries #1}]}{\end{trivlist}}
%
%\newcommand{\qed}{\nobreak \ifvmode \relax \else
%      \ifdim\lastskip<1.5em \hskip-\lastskip
%      \hskip1.5em plus0em minus0.5em \fi \nobreak
%      \vrule height0.75em width0.5em depth0.25em\fi}

%\usepackage{eso-pic}

%\usepackage[pdftex,colorlinks = true,linkcolor = blue,urlcolor = blue, citecolor = blue]{hyperref}

\usepackage[sort]{natbib} \input{natbib-add}
\bibliographystyle{named-wits}
\bibpunct{(}{)}{;}{a}{}{}  % to get correct punctuation for bibliography
\setlength{\skip\footins}{1.5cm}
\newcommand{\citets}[1]{\citeauthor{#1}'s \citeyearpar{#1}}

\renewcommand\bibname{References}  % change title of references section

%\newcommand{\myquote}[2]{\begin{center}\fbox{\begin{minipage}{.95\linewidth}#1\hfill #2\end{minipage}}\end{center}}

%
%\makeatletter\newenvironment{graybox}{%
%   \begin{lrbox}{\@tempboxa}\begin{minipage}{0.8\linewidth}}{\end{minipage}\end{lrbox}%
%   \colorbox{lightgray}{\usebox{\@tempboxa}}
%}\makeatother
%
%\makeatletter\newenvironment{mygraybox}{%
%   \begin{lrbox}{\@tempboxa}\begin{minipage}{0.95\linewidth}}{\end{minipage}\end{lrbox}%
%   \colorbox{lightgray}{\usebox{\@tempboxa}}
%}\makeatother


\usepackage{color}
\definecolor{quotationcolour}{gray}{0.8}
\definecolor{quotationmarkcolour}{gray}{0.4}
\definecolor{lightgray}{gray}{.8}

% Double-line for start and end of epigraph.
\newcommand{\epiline}{\hrule \vskip -.2em \hrule}
% Massively humongous opening quotation mark.
\newcommand{\hugequote}{%
  \fontsize{42}{48}\selectfont \color{quotationmarkcolour} \textbf{``}
  \vskip -.5em
}

\newcommand*{\signed}[1]{%
  \unskip\hspace*{1em plus 1fill}%
  \nolinebreak[3]\hspace*{\fill}\mbox{#1}
}

% Beautify quotations.
\newcommand{\epigraph}[2]{%
  %\bigskip
  \begin{center}
  \colorbox{quotationcolour}{%
    \parbox{.90\textwidth}{%
    \epiline \vskip 1em {\hugequote} \vskip -.5em
    \parindent 2.2em
    #1\signed{#2}\\
    \epiline
    }
  }
  \end{center}
  %\bigskip
}

\usepackage{environ}
\NewEnviron{myquote}[1]{\epigraph{\BODY}{#1}}
\usepackage{lettrine}

\usepackage{listing}
\usepackage{minted}

\newcommand{\inputcode}[1]{\inputminted[bgcolor=white,linenos=false,fontfamily=courier,frame=single]{c}{#1}}
\newcommand{\ccode}{\mint[mathescape,bgcolor=white,linenos=false,fontfamily=courier,frame=none]{c}}

\definecolor{codebg}{rgb}{0.95,0.95,0.95}
\newmintedfile[makecode]{make}{tabsize=4, fontsize=\small, frame=lines, linenos, framesep=\fboxsep, bgcolor=codebg, rulecolor=\color{gray!40},samepage=true}
\newmintedfile[cppcode]{cpp}{tabsize=4, fontsize=\small, frame=lines, mathescape, linenos, mathescape, framesep=\fboxsep, bgcolor=codebg, rulecolor=\color{gray!40},samepage=false}
\newminted[cpp]{cpp}{tabsize=4, fontsize=\small, frame=lines, linenos, mathescape, framesep=\fboxsep, bgcolor=codebg, rulecolor=\color{gray!40},samepage=false}
\newminted[bash]{bash}{tabsize=4, fontsize=\small, frame=lines, linenos=false, mathescape, framesep=\fboxsep, bgcolor=codebg, rulecolor=\color{gray!40},samepage=true}
\newminted[sql]{sql}{tabsize=4, fontsize=\small, frame=lines, linenos=false, mathescape, framesep=\fboxsep, bgcolor=codebg, rulecolor=\color{gray!40},samepage=true}

\usepackage{mathtools}
\usepackage{colortbl}
\usepackage{tikz}
\usetikzlibrary{positioning}
\usetikzlibrary{decorations.text}
\usetikzlibrary{decorations.pathmorphing}
\makeatletter
\tikzset{
	block filldraw/.style={% only the fill and draw styles
		draw, fill=none},
	block rect/.style={% fill, draw + rectangle (without measurements)
		block filldraw, rectangle},
	block/.style={% fill, draw, rectangle + minimum measurements
		block rect, minimum height=0.8cm, minimum width=6em},
	from/.style args={#1 to #2}{% without transformations
		above right={0cm of #1},% needs positioning library
		/utils/exec=\pgfpointdiff
		{\tikz@scan@one@point\pgfutil@firstofone(#1)\relax}
		{\tikz@scan@one@point\pgfutil@firstofone(#2)\relax},
		minimum width/.expanded=\the\pgf@x,
		minimum height/.expanded=\the\pgf@y}}
\makeatother

\theoremstyle{plain}
\newtheorem{thm}{Theorem}[chapter] % reset theorem numbering for each chapter

\theoremstyle{definition}
\newtheorem{layer}[thm]{Layer} 


%\usepackage{todonotes}
\setlength{\marginparwidth}{2cm}

\usepackage{nomencl}
\makenomenclature

\usepackage{lscape}
\usepackage{pdflscape}
\usepackage{afterpage}
\usepackage{capt-of}% or use the larger `caption` package

% Blank line between paragraphs, but keep the paragraph indent
\edef\restoreparindent{\parindent=\the\parindent\relax}
\usepackage{parskip}
\restoreparindent

\usepackage{adjustbox}
\usepackage{longtable}
